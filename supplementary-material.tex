\listfiles
\documentclass[12pt]{article}
% -------------------------------------------------------------------
% Basic packages
\usepackage[english]{babel}
\usepackage[utf8]{inputenc}
\usepackage[T1]{fontenc}
\usepackage{lineno}
\usepackage{hyperref}
\usepackage{graphicx}
% -------------------------------------------------------------------
% Math packages
\usepackage{amsmath,amsfonts,amssymb,amsthm,cancel,siunitx,calculator,calc,mathtools,empheq,latexsym,inputenc}
% -------------------------------------------------------------------
% Packages for figures
\usepackage{subfig,epsfig,tikz,float}
% -------------------------------------------------------------------
% Packages for tables
\usepackage{booktabs,multicol,multirow,tabularx,array}
%\usepackage{natbib}
% -------------------------------------------------------------------

% Margins

\usepackage[a4paper,
            bindingoffset=0.2in,
            left=1in,
            right=1in,
            top=1in,
            bottom=1in,
            footskip=.25in]{geometry}


% Begin document


\begin{document}

\title{Supplementary material for: \\
Geographic and seasonal variation of the \textit{for} gene reveal signatures of local adaptation in \textit{Drosophila melanogaster}}
\vspace{25px}
\date{}
\maketitle

\vspace{30px}

\section*{Tables}

\setcounter{table}{0}
\renewcommand{\thetable}{S\arabic{table}}


% latex table generated in R 4.2.2 by xtable 1.8-4 package
% Sun Jun 18 21:30:16 2023
\begin{table}[ht]
\tiny
\centering
\caption*{\tiny Table S1: Number of seasons surveyed across localities. Abbreviations are as follows: Aus: Austria, Charl = Charlottesville, Denm = Denmark, Esp = Esparto, Fin = Finland, Fran = France, Germ = Germay, PA = Pennsylvania; MI-WI = Michigan and Wisconsin; NY-MA = New York and Massachusetts, Tuol = Tuolume, Ukr = Ukraine.}
\begin{tabular}{llllllllllllllll}
  \hline
 & Aus & Charl & Denm & Esp & Fin & Fran & Germ & MI-WI & NY-MA & PA & Russia & Spain & Tuol & Turkey & Ukr \\ 
  \hline
fall &   3 &   3 &   4 &   3 &   2 &   4 &   6 &   5 &   3 &   9 &   3 &   4 &   4 &   4 &  14 \\ 
  spring &   3 &   3 &   1 &   2 &   4 &   6 &   6 &   4 &   3 &   8 &   3 &   3 &   1 &   9 &  24 \\ 
   \hline
\end{tabular}
\end{table}


\end{document}





\end{document}